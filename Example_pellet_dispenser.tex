\section{Example using pellet
dispenser}\label{example-using-pellet-dispenser}

This notebook contains an example of how to run the code for using the
multimaterial dispenser. Most cells can be run without any modification
if the dispenser has been built without design modifications. The lines
that need modifications are marked like this:

\passthrough{\lstinline!\#\#\#\#\#\#\#\#\#\#\#\#\#\#\#\#\#\#\#\#\#\#\#\#\#\#\#\#\#\#\#\#\#\#\#\#\#\#\#\#\#\#\#\#\#\#\#\#\#\#\#\#\#\#\#\#\#\#\#\#\#\#\#\#\#\#\#\#\#\#\#\#\#\#\#\#\#\#\#!}

\begin{lstlisting}[language=Python]
# Import Libraries
import serial
import numpy as np
import time
import serial.tools.list_ports
import os
import sys
sys.path.append('src')
from DispenserSetup import SetupDispenser, Compositions
\end{lstlisting}

In this cell, we define the port name for the arduino and the balance.
We do it by listing the connected ports and selecting the ones whose
device serial is the corresponding to our device. You should find the
serial of your devices and change it accordingly. You can do it by
inspecting the devices detected by serial.tools.list\_ports.comports().
In case of doubt, try disconnecting and connecting to see which ports
are affected.

\begin{lstlisting}[language=Python]
ports = serial.tools.list_ports.comports()

###############################################################################

arduino_serial = "write_here_your_arduino_serial"
balance_serial = "write_here_your_balance_serial"

###############################################################################

for port in ports:
    if port.serial_number == arduino_serial:
        arduino_port = port.device
    elif port.serial_number == balance_serial:
        balance_port = port.device
        
### Serial initialization
## Arduino
arduino = serial.Serial(arduino_port, 9600)
## Balance
balance = serial.Serial(port=balance_port, baudrate=9600, bytesize=serial.SEVENBITS,
                       parity=serial.PARITY_EVEN, stopbits=serial.STOPBITS_ONE)
\end{lstlisting}

In the next cell, we use the Arduino Command Line Interface
(arduino-cli) to compile and upload the arduino script. If you prefer to
do it using the Graphical User Interface, skip this cell. For using it,
update the location of your arduino-cli

\begin{lstlisting}[language=Python]
###############################################################################################################

os.system("PATH/arduino-cli compile --fqbn arduino:avr:mega 8noIRnew_NOshaketimer_new/8noIRnew_NOshaketimer_new.ino")
os.system("PATH/arduino-cli upload -p "+arduino_port+" --fqbn arduino:avr:mega 8noIRnew_NOshaketimer_new/8noIRnew_NOshaketimer_new.ino")

###############################################################################################################
\end{lstlisting}

The next step is to define the materials and their concentrations in
each of the dispensers. For that we will use the
\passthrough{\lstinline!SetupDispenser!} function from
\passthrough{\lstinline!/src/DispenserSetup!}. It will require a
\passthrough{\lstinline!.csv!} file detailing the compositions, that can
be built following the example available in the help of the function:

\begin{lstlisting}[language=Python]
help(SetupDispenser)
\end{lstlisting}

\begin{lstlisting}[language=Python]
###############################################################################
D, M, A = SetupDispenser("data/dispenser_setup.csv", "PLA")
###############################################################################
\end{lstlisting}

Next, we read the file that contains the compositions using the
Compositions function. It's working is detailed in the documentation:

\begin{lstlisting}[language=Python]
help(Compositions)
\end{lstlisting}

\begin{lstlisting}[language=Python]
###############################################################################
compositions = Compositions("./data/compositions_to_dispense.csv", D, M, A, "PLA")
###############################################################################
print(compositions)
\end{lstlisting}

For the rest of the notebook we will use only the first of the
compositions available in the .csv file. in your application you can
easily loop over all of them, just remember to change the cup in the
balance between compositions.

\begin{lstlisting}[language=Python]
C = compositions[0]
\end{lstlisting}

It is convenient now to establish a ratio between materials: in this
way, instead of dispensing all the ammount of one material and then of
another, we can alternate small quantities and thus have a better
mixing. We do this by selecting the one with lower ammount and
expressing the others as a ratio with respect to this one. In every step
we will dispense 1 g of the smaller one and the corresponding grams of
the rest.

\begin{lstlisting}[language=Python]
tol = 1e-3
minC = np.min([c for c in C if c>tol])
ratios = C/minC 
print('The ratios for pellet dispensing will be: ', ratios)
\end{lstlisting}

The next cell stablish commands to tare, calibrate and measure weight
from the balance. You may need to edit them if your balance uses a
different protocol.

\begin{lstlisting}[language=Python]
###############################################################################
def tare(balance):
    # Set the balance measurement to 0.0 g
    try:
        balance.write(b'T\r\n')
    except:
        balance = serial.Serial(port=balance_port, baudrate=9600, bytesize=serial.SEVENBITS,
                       parity=serial.PARITY_EVEN, stopbits=serial.STOPBITS_ONE)
        balance.write(b'T\r\n')


def calib(balance):
    # Calibrate the balance automatically
    try:
        balance.write(b'C\r\n')
    except:
        balance = serial.Serial(port=balance_port, baudrate=9600, bytesize=serial.SEVENBITS,
                       parity=serial.PARITY_EVEN, stopbits=serial.STOPBITS_ONE)
        balance.write(b'C\r\n')

def measure(balance):
    try:
        trash = balance.read_all()
        balance.write(b'B\r\n')
        reading = balance.readline()
        weight = float(reading[:10])
    except:
        balance = serial.Serial(port=balance_port, baudrate=9600, bytesize=serial.SEVENBITS,
                       parity=serial.PARITY_EVEN, stopbits=serial.STOPBITS_ONE)
        trash = balance.read_all()
        balance.write(b'B\r\n')
        reading = balance.readline()
        weight = float(reading[:10])
    return weight
###############################################################################
\end{lstlisting}

Select the total ammount of material that you want to dispense

\begin{lstlisting}[language=Python]
###############################################################################
mass_target = 100 # g
###############################################################################
\end{lstlisting}

Now, we define some variables and arrays that we will need in the
dispensing loop

\begin{lstlisting}[language=Python]
measureAll = True
massAll = 0
mass = 0
masses = [[0] for _ in range(8)]
i = 0
times = [[0] for _ in range(8)]


dispenser_switch = {0 : b'0', 1 : b'1', 2 : b'2', 3 : b'3', 
                    4 : b'4', 5 : b'5', 6 : b'6', 7 : b'7', 'stop' : b'8'}
\end{lstlisting}

The next cell runs the dispensers until the desired quantity is reached.
As pointed above, it will alternate dispensers keeping the ratios
between materials.

\begin{lstlisting}[language=Python]
start = time.time()

while measureAll == True:
    for j in range(len(C)):
        if C[j] > tol:
            if masses[j][i] < ratios[j]*(i+1):
                measureNow = True
                tare(balance)
                time.sleep(2)
                arduino.write(dispenser_switch[j])
                while measureNow == True:
                    mass = measure(balance)
                    if mass + masses[j][i] >= ratios[j]*(i+1):
                        arduino.write(dispenser_switch['stop'])
                        time.sleep(4)
                        mass = measure(balance)
                        measureNow = False
                massAll = massAll + mass
                masses[j].append(masses[j][i]+mass)
                times[j].append(time.time()-start)
                time.sleep(0.1)
            elif masses[j][i] >= ratios[j]*(i+1):
                arduino.write(dispenser_switch['stop'])
            print("Dispenser ", j, "| iteration: ", i+1, "| t = ", times[j][i+1], "s | ", "mass : ", masses[j][i+1])                  
    if massAll > mass_target:
        measureAll = False
        arduino.write(dispenser_switch['stop'])
    i = i + 1
    
arduino.write(dispenser_switch['stop'])
\end{lstlisting}

We can now plot the dispensing progress, comparing the amount that was
actually dispensed and the one asked by the code

\begin{lstlisting}[language=Python]
import matplotlib.pyplot as plt

colors = ["C0", "C1", "C2", "C3", "C4", "C5", "C6", "C7"] 

for key in D.keys():
    plt.plot(times[D[key]-1], masses[D[key]-1],label=key, c=colors[D[key]-1])
    plt.plot(times[D[key]-1], np.arange(len(times[D[key]-1]))*ratios[D[key]-1], '--',c=colors[D[key]-1])
plt.plot(0,0,'--',c = 'gray',label = "Expected amounts")
plt.title("Progression of Sample")
plt.xlabel("Time [s]")
plt.ylabel("Mass [g]")
plt.legend()
plt.show()
\end{lstlisting}

Finally, we present a measure for accuracy consistent of the total error
commited divided by the total amount dispensed:

\begin{lstlisting}[language=Python]
final_masses = [mass[-1] for mass in masses[:]]
accuracy = (1-np.sum(abs(final_masses-C*mass_target))/np.sum(final_masses))*100
print("Accuracy: ", accuracy, "%")
\end{lstlisting}

\begin{lstlisting}[language=Python]
\end{lstlisting}
